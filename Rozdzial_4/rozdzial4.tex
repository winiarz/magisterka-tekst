
\section { Implementacja dla GPU - ułożenie danych w pamięci }
Dane dla algorytmu można dostarczyć w różny sposób:
\subsection { Tablica struktur }
Parametry pojedynczego punktu ułożone są w strukturę, a całe strukturey w tablicę. Takie podejście sprawia, że OpenCL zaokrągla rozmiar struktury w górę do pełnych 16 bajtów tracąc przy tym 12,5 \% pamięci.
\subsection { Struktura tablic }
Współrzędne położenia x wszystkich punktów przechowywane są w jednej tablicy, współrzędne y w osobnej. Analogicznie współrzędne położenia z, kolejne współrzędne wektorów prędkości oraz masy punktów. 
Porównanie wydajności tego samego algorytmu przy różnym ułożeniu danych przedstawia tabela:

\subsection {Wykorzystanie operacji wektorowych}

\subsection { Porównanie }
