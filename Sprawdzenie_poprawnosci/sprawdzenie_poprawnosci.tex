\section { Sprawdzanie poprawności algorytmu - czy warto o tym pisać? }

Wzorcowy algorytm (wersja na CPU) został sprawdzony w następujący sposób:

Test 1 - wszystkie punkty w tym samym miejscu.
Test 2 - punky na okręgu powinny się poruszać do środka
Test 3 - druga prędkość kosmiczna

Ze względu na błędy zaokrągleń wynik jest uznawany za poprawny, jeśli nie odbiega od wzorcowego (wersji CPU) nie więcej niż z góry zadany $\epsilon$. Błędy zaokrągleń są trudne do oszacowania, przez co wartość $\epsilon$ została ustalona w sposób eksperymentaly na $\epsilon = 0.05$.
