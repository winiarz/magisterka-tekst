
\subsubsection {Metoda Eulera}

Metoda Eulera służy do rozwiązywania równań różniczkowych pierwszego rzędu, czyli:

$$x' = f(x)$$

Rozwiązanie takiego równania przybliża się za pomocą dwóch pierwszych wyrazów szeregu Taylora:

$$x(t+\Delta t) = x(t) + \Delta t \cdot f(x)$$

W przypadku równań ruchu wystarczy zastosować tą metodę dwukrotnie. Pierwszy raz do obliczenia prędkości za pomocą przyspieszenia. Drugi do obliczenia położenia używając prędkości z pierwszego kroku. Całość sprowadza się do wzorów:

\begin{center}

$
x( t_{0} + \Delta t) = x(t_{0}) + x'(t_{0}) \cdot \Delta t
$

$
x'( t_{0} + \Delta t) = x'(t_{0}) + x''(t_{0}) \cdot \Delta t
$
\end{center}

%Jest to jedna z najprostszych metod, wymaga jednorazowego obliczenia przyspieszenia na krok czasowy.
%Jej błąd jest rzędu $ O( \Delta t ^{2} ) $,