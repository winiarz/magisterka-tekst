
\subsubsection {Metoda punktu środkowego}

Metoda punktu środkowego składa się z dwóch etapów. W pierwszym obliczany jest tzw. punkt środkowy. Jest to punkt leżący na środku odcinka łączącego obecne położenie z przybliżonym położeniem w następnym kroku czasowym, które wyznaczane jest za pomocą metody Eulera. W punkcie środkowym ponownie obliczane jest przyspieszenie, które jest używane do wyznaczenia następnego położenia za pomocą metody Eulera.

\begin{center}

$
k_{1v} = \Delta t \cdot x''( t_{0} )
$

$
k_{1x} = \Delta t \cdot x' ( t_{0} )
$

$
k_{2v} = \Delta t \cdot
x''( t_{0} + \frac {\Delta t \cdot k_{1} } {2} )
$

$
k_{2x} = \Delta t \cdot \frac {1} {2} \cdot k_{1v}
$

$
x'( t_{0} + \Delta t ) = x'( t_{0} ) + k_{2v}
$

$
x( t_{0} + \Delta t ) = x( t_{0} ) + k_{2x}
$

\end{center}

%Metoda punktu środkowego jest dokładniejsza od metody Eulera, jej błąd jest rzędu $ O( \Delta t ^{3} ) $
%Jednak w przypadku problemu N-Ciał metoda ta jest bardzo niewydajna. Wymaga ona bowiem dwukrotnego wykonywania obliczeń przyspieszeń punktów, a zatem odległości między nimi, co stanowi najdłuższy etap obliczeń.