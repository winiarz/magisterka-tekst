
\section { Problem N-Cial }
\subsection {Symulowane zjawiska fizyczne}
Problem n-ciał to zagadnienie mechaniki klasycznej polegające na obliczeniu trajektorii ruchu skończonej liczby punktów materialnych. 
Każdy z tych punktów oddziałuje na każdy inny siłą zadaną wzorem:

\begin{center}

$ F = G \cdot \frac{m_{1} \cdot m_{2} \cdot ( \overrightarrow{x_{1}} - \overrightarrow{x_{2}}) } 
{ || \overrightarrow{x_{1}} - \overrightarrow{x_{2}} || ^{3} } $ 
\end{center}

czyli prawem powszechnego ciążenia Izaaka Newtona.\linebreak
Przykładami takich problemów mogą być symulacja galaktyki, układu planetarnego itp.
Ze względu na podobieństwo (wzorów)/(praw fizyki) używając tego samego algorytmu można symulować cząstki oddziaływujęce na siebie siłą elektrostatyczną.\linebreak
Obliczenie trajektorii wymaga rozwiązania układu równań różniczkowych drugiego rzędu:
\begin{center}

$ x_{i}''  = G \sum\limits_{j=1}^{n} m_ {j} \frac 
{\overrightarrow{x_{1}} - \overrightarrow{x_{2}}}
{||\overrightarrow{x_{1}} - \overrightarrow{x_{2}}|| ^{3}} $  gdzie i = 1 .. n 
\end{center}


Do tej pory nie rozwiązano tego układu w sposób analityczny dla $n > 2$, dlatego rozwiązanie przybliża się stosując metody numeryczne.

\subsection {Metody symulowania punktów materialnych}

Znając początkowe położenie, prędkość oraz przyspieszenie ruch punktu materialnego można symulować na kilka sposobów:

\subsubsection {Metoda Eulera}
Metoda Eulera polega na przybliżeniu rozwiązania za pomocą szeregu Taylor'a w którym zachowano tylko pierwszą pochodną.
W przypadku równań ruchu mechaniki klasycznej metoda Eulera sprowadza się  do wzorów:

\begin{center}

$
x( t_{0} + \Delta t) = x(t_{0}) + x'(t_{0}) \cdot \Delta t
$

$
x'( t_{0} + \Delta t) = x'(t_{0}) + x''(t_{0}) \cdot \Delta t
$
\end{center}

Jest to jedna z najprostszych metod, wymaga jednorazowego obliczenia przyspieszenia na krok czasowy.
Jej błąd jest rzędu $ O( \Delta t ^{2} ) $,

\subsubsection {Metoda punktu środkowego}

$
k_{1v} = \Delta t \cdot x''( t_{0} )
$

$
k_{1x} = \Delta t \cdot x' ( t_{0} )
$

$
k_{2v} = \Delta t \cdot
x''( t_{0} + \frac {\Delta t \cdot k_{1} } {2} )
$

$
k_{2x} = \Delta t \cdot \frac {1} {2} \cdot k_{1v}
$

$
x'( t_{0} + \Delta t ) = x'( t_{0} ) + k_{2v}
$

$
x( t_{0} + \Delta t ) = x( t_{0} ) + k_{2x}
$


\subsubsection {Metoda Verleta}

$
x(t_{0} + \Delta t) =
2 \cdot x(t_{0}) -
x(t_{0} - \Delta t) +
x'' (t_{0}) \cdot 
\Delta t ^{2}
$

