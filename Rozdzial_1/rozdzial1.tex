
\section { Problem N-Cial }
\subsection {Symulowane zjawiska fizyczne}
Problem n-ciał to zagadnienie mechaniki klasycznej polegające na obliczeniu trajektorii ruchu skończonej liczby punktów materialnych. 
Każdy z tych punktów oddziałuje na każdy inny siłą zadaną wzorem:

\begin{center}

$ F = G \cdot \frac{m_{1} \cdot m_{2} \cdot ( \overrightarrow{x_{1}} - \overrightarrow{x_{2}}) } 
{ || \overrightarrow{x_{1}} - \overrightarrow{x_{2}} || ^{3} } $ 
\end{center}

czyli prawem powszechnego ciążenia Izaaka Newtona.\linebreak
Przykładami takich problemów mogą być symulacja galaktyki, układu planetarnego itp.
Ze względu na podobieństwo (wzorów)/(praw fizyki) używając tego samego algorytmu można symulować cząstki oddziaływujęce na siebie siłą elektrostatyczną.\linebreak
Obliczenie trajektorii wymaga rozwiązania układu równań różniczkowych drugiego rzędu:
\begin{center}

$ x_{i}''  = G \sum\limits_{j=1}^{n} m_ {j} \frac 
{\overrightarrow{x_{1}} - \overrightarrow{x_{2}}}
{||\overrightarrow{x_{1}} - \overrightarrow{x_{2}}|| ^{3}} $  gdzie i = 1 .. n 
\end{center}


Do tej pory nie rozwiązano tego układu w sposób analityczny dla $n > 2$, dlatego rozwiązanie przybliża się stosując metody numeryczne. Istnieje kilka metod pozwalających na przybliżanie kolejnych pozycji punktu materialnego. Oto najważniejsze z nich:

\subsection {Metody symulowania punktów materialnych}

Znając początkowe położenie, prędkość oraz przyspieszenie ruch punktu materialnego można symulować na kilka sposobów:

\subsubsection {Metoda Eulera}

Metoda Eulera służy do rozwiązywania równań różniczkowych pierwszego rzędu, czyli:

$$x' = f(x)$$

Rozwiązanie takiego równania przybliża się za pomocą dwóch pierwszych wyrazów szeregu Taylora:

$$x(t+\Delta t) = x(t) + \Delta t \cdot f(x)$$

W przypadku równań ruchu wystarczy zastosować tą metodę dwukrotnie. Pierwszy raz do obliczenia prędkości za pomocą przyspieszenia. Drugi do obliczenia położenia używając prędkości z pierwszego kroku. Całość sprowadza się do wzorów:

\begin{center}

$
x( t_{0} + \Delta t) = x(t_{0}) + x'(t_{0}) \cdot \Delta t
$

$
x'( t_{0} + \Delta t) = x'(t_{0}) + x''(t_{0}) \cdot \Delta t
$
\end{center}

%Jest to jedna z najprostszych metod, wymaga jednorazowego obliczenia przyspieszenia na krok czasowy.
%Jej błąd jest rzędu $ O( \Delta t ^{2} ) $,

\subsubsection {Metoda punktu środkowego}
Metoda punktu środkowego jest modyfikacją metody Eulera. Najpierw stosujemy metodę Eulera, tylko ze zmiejszonym o połowę krokiem czasowym. Uzyskany wynik nazywamy ``punktem środkowym''. Następnie obliczamy przyspieszenie w tym punkcie. Położenie oraz przyspieszenie punktu aktualizujemy używając punktu startowego i przyspieszenia w punkcie środkowym.

\begin{center}

$
k_{1v} = \Delta t \cdot x''( t_{0} )
$

$
k_{1x} = \Delta t \cdot x' ( t_{0} )
$

$
k_{2v} = \Delta t \cdot
x''( t_{0} + \frac {\Delta t \cdot k_{1} } {2} )
$

$
k_{2x} = \Delta t \cdot \frac {1} {2} \cdot k_{1v}
$

$
x'( t_{0} + \Delta t ) = x'( t_{0} ) + k_{2v}
$

$
x( t_{0} + \Delta t ) = x( t_{0} ) + k_{2x}
$

\end{center}

 ??? Coś o metodach Rungego-Kutty ???
Metoda punktu środkowego jest dokładniejsza od metody Eulera, jej błąd jest rzędu $ O( \Delta t ^{3} ) $
Jednak w przypadku problemu N-Ciał metoda ta jest bardzo niewydajna. Wymaga ona bowiem dwukrotnego wykonywania obliczeń przyspieszeń punktów, a zatem odległości między nimi, co stanowi najdłuższy etap obliczeń.

\subsubsection {Algorytm Verleta}
Algorytm Verleta opiera się na rozwinięciu funkcji położenia w szereg Taylora do trzeciej pochodnej dla kroku czasowego do przodu i w tył.
\linebreak
Tu będą obliczenia
\linebreak
Po prostych przekształceniach otrzymujemy:
\begin{center}

$
x(t_{0} + \Delta t) =
2 \cdot x(t_{0}) -
x(t_{0} - \Delta t) +
x'' (t_{0}) \cdot 
\Delta t ^{2}
$

\end{center}

Obliczenie następnego położenia w czasie nie wymaga obliczania prędkości punktu, a jedynie znajomości jego 2 poprzednich położeń. 
Obliczenie prędkości nie jest jednak wymagane w przypadku problemu N-Ciał, gdyż siły zależą tylko od obecnego położenia punktów. 
\linebreak

W przypadku problemu N-Ciał najbardziej czasochłonną część obliczeń stanowi obliczenie sił między punktami. Metody Rungego-Kutty zostały odrzucone ze względu na konieczność kilkukrotnego powtarzania tej części obliczeń. Metoda Verleta jest najdokladniejsza spośród pozostałych i dlatego na potrzeby tej zastosowana została na potrzeby tej pracy.
