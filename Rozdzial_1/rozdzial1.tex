
\section { Problem N-Cial }


\subsection {Symulowane zjawiska fizyczne}
Problem n-ciał to zagadnienie mechaniki klasycznej polegające na obliczeniu trajektorii ruchu skończonej liczby punktów materialnych. 
Każdy z tych punktów oddziałuje na każdy inny siłą zadaną wzorem:

\begin{center}

$ F = G \cdot \frac{m_{1} \cdot m_{2} \cdot ( \overrightarrow{x_{1}} - \overrightarrow{x_{2}}) } 
{ || \overrightarrow{x_{1}} - \overrightarrow{x_{2}} || ^{3} } $ 
\end{center}

czyli prawem powszechnego ciążenia Izaaka Newtona.\linebreak
Przykładami takich problemów są głównie symulacje ciał niebieskich: układów planetarncyh, galaktyk, wszechświata, lecz ze względu na analogię praw fizyki używając w ten sam sposób można symulować cząstki oddziaływujęce na siebie siłą elektrostatyczną.\linebreak

Obliczenie trajektorii wymaga rozwiązania układu równań różniczkowych drugiego rzędu:
\begin{center}

$ x_{i}''  = G \sum\limits_{j=1}^{n} m_ {j} \frac 
{\overrightarrow{x_{1}} - \overrightarrow{x_{2}}}
{||\overrightarrow{x_{1}} - \overrightarrow{x_{2}}|| ^{3}} $  gdzie i = 1 .. n 
\end{center}


Do tej pory nie rozwiązano tego układu w sposób analityczny dla $n > 2$, dlatego rozwiązanie przybliża się stosując metody numeryczne. 



\subsection {Metody symulowania punktów materialnych}

Istnieje kilka metod numerycznych pozwalających na rozwiązanie równań mechaniki klasycznej, czyli obliczanie położeń oraz prędkości punktów materialnych w kolejnych krokach czasowych.

Wszystkie wymagają podania warunków brzegowych, którymi są początkowe położenia i prędkości wszystkich punktów materialnych. Metody te różnią się od siebie jedynie dokładnością obliczeń oraz szybkością działania. Poniżej przedstawiam najważniejsze z nich. \linebreak


\subsubsection {Metoda Eulera}
Metoda Eulera polega na przybliżeniu rozwiązania za pomocą szeregu Taylor'a w którym zachowano tylko pierwszą pochodną.
W przypadku równań ruchu mechaniki klasycznej metoda Eulera sprowadza się  do wzorów:

\begin{center}

$
x( t_{0} + \Delta t) = x(t_{0}) + x'(t_{0}) \cdot \Delta t
$

$
x'( t_{0} + \Delta t) = x'(t_{0}) + x''(t_{0}) \cdot \Delta t
$
\end{center}

Jest to jedna z najprostszych metod, wymaga jednorazowego obliczenia przyspieszenia na krok czasowy.
Jej błąd jest rzędu $ O( \Delta t ^{2} ) $,

\subsubsection {Metoda punktu środkowego}

Metoda punktu środkowego składa się z dwóch etapów. W pierwszym obliczany jest tzw. punkt środkowy. Jest to punkt leżący na środku odcinka łączącego obecne położenie z przybliżonym położeniem w następnym kroku czasowym, które wyznaczane jest za pomocą metody Eulera. W punkcie środkowym ponownie obliczane jest przyspieszenie, które jest używane do wyznaczenia następnego położenia za pomocą metody Eulera.

\begin{center}

$
k_{1v} = \Delta t \cdot x''( t_{0} )
$

$
k_{1x} = \Delta t \cdot x' ( t_{0} )
$

$
k_{2v} = \Delta t \cdot
x''( t_{0} + \frac {\Delta t \cdot k_{1} } {2} )
$

$
k_{2x} = \Delta t \cdot \frac {1} {2} \cdot k_{1v}
$

$
x'( t_{0} + \Delta t ) = x'( t_{0} ) + k_{2v}
$

$
x( t_{0} + \Delta t ) = x( t_{0} ) + k_{2x}
$

\end{center}

%Metoda punktu środkowego jest dokładniejsza od metody Eulera, jej błąd jest rzędu $ O( \Delta t ^{3} ) $
%Jednak w przypadku problemu N-Ciał metoda ta jest bardzo niewydajna. Wymaga ona bowiem dwukrotnego wykonywania obliczeń przyspieszeń punktów, a zatem odległości między nimi, co stanowi najdłuższy etap obliczeń.

% ??? Coś o metodach Rungego-Kutty ???

\subsubsection {Metody Rungego-Kutty czwartego rzędu}

Metoda punktu środkowego jest właściwie, metodą Rungego-Kutty drugiego rzędu. Metoda czwartego rzędu jest jej rozszerzeniem o 2 kolejne kroki. Każdy kolejny krok jest zastosowaniem metody Eulera z coraz dokładniejszym przyspieszeniem. Niestety każdy kolejny krok wymaga ponownego obliczenia przyspieszenia. /linebreak

?? wstawić tu wzor - chyba tak







\subsection{Wybór metody dla problemu N-Ciał}
W przypadku problemu N-Ciał najbardziej czasochłonną część obliczeń stanowi obliczenie sił między punktami. Metody Rungego-Kutty zostały odrzucone ze względu na konieczność kilkukrotnego powtarzania tej części obliczeń. Metoda Verleta jest najdokladniejsza spośród pozostałych i dlatego na potrzeby tej zastosowana została na potrzeby tej pracy.
