
\section { Problem N-Cial }
\subsection {Symulowane zjawiska fizyczne}
Problem n-ciał to zagadnienie mechaniki klasycznej polegające na obliczeniu trajektorii ruchu skończonej liczby punktów materialnych. 
Każdy z tych punktów oddziałuje na każdy inny siłą zadaną wzorem:

\begin{center}

$ F = G \cdot \frac{m_{1} \cdot m_{2} \cdot ( \overrightarrow{x_{1}} - \overrightarrow{x_{2}}) } 
{ || \overrightarrow{x_{1}} - \overrightarrow{x_{2}} || ^{3} } $ 
\end{center}

czyli prawem powszechnego ciążenia Izaaka Newtona.\linebreak
...
Obliczenie trajektorii wymaga rozwiązania układu równań różniczkowych:
\begin{center}

$ x_{i}''  = G \sum\limits_{j=1}^{n} m_ {j} \frac 
{\overrightarrow{x_{1}} - \overrightarrow{x_{2}}}
{||\overrightarrow{x_{1}} - \overrightarrow{x_{2}}|| ^{3}} $  gdzie i = 1 .. n 
\end{center}

\subsection {Sposoby symulowania punktów materialnych}
