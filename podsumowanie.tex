
\section{Podsumowanie (jak lepiej to nazwać?)}

%Ze względu na różnorodność kart graficznych niemożliwe jest stworzenie jednego algorytmu optymalnego dla nich wszystkich.
Karty graficzne chociaż ich architektury są zbliżone to jednak różnią się szczegółami technicznymi. Poszczególne model ma inne rozmiary pamięci globalnej, współdzielonej jak i stałej.
Mają inną maksymalną liczbę wątków, pogrupowaną w inne grupy w zależności od producenta ( nazywane WARP w przypadku NVIDIA, lub ... w przypadku AMD ). Przyspieszają sprzętowo różne operacje. Te różnice sprawiają, że na każdej karcie inna implementacja może okazać się tą optymalną. \linebreak


Standard OpenCL daje tylko możliwość uruchomienia algorytmu na wielu platformach sprzętowych. Jednak znajomość sprzętu na jakim kernel będzie wykonywany dalej jest niezbędna do optymalizacji. 
Uniemożliwia to napisanie programu, który działa optymalnie na każdym sprzęcie. Gdyby udało się to osiągnąć możnaby pisać komercyjne aplikacje dzilające na sprzęcie klienta bez znajomości tego sprzętu. \linebreak
% coś o komercyjnym zastosowaniu



%Widać to na przykładzie tej pracy. Zastosowane optymalizacje mają zupełnie inny wpływ na wydajnopść algorytmu w zależności od modelu GPU. 

OpenCl daje jednak możliwość odczytania wielu parametrów specyfikacji użądzenia. Można wykorzystać je jako parametry algorytmu. Najprostszym tego przykładem jest tradycyjna równoległa pętla for. Parametrem w tym przypadku jest maksymalna liczba wątków dostępnych na użądzeniu.
\linebreak

\lstinputlisting[language=C]{Source/petla_for.c}

Tak jak w przykładzie tej pracy ilość jednocześnie przetwarzanych punktów została uzależniona od dostępnej pamięci prywatnej i współdzielonej.
\linebreak

W innych przypadkach o wyborze optymalnej implementacji może decydować np. dostępność operacji wektorowych o odpowiednim rozmiarze, przyspieszenie sprzętowe dla funkcji takich jak sqrt, sin, log.

Sparametryzowany w ten sposób algorytm może być wykorzystany w oprogramowaniu uruchamianym na komputerze klienta. Wtedy bez względu na posiadany przez niego sprzęt będzie działało wydajnie.
\linebreak



