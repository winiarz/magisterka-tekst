
\section{Podsumowanie (jak lepiej to nazwać?)}

Ze względu na różnorodność kart graficznych niemożliwe jest stworzenie jednego algorytmu optymalnego dla nich wszystkich. Jek widać na przykładzie optymalizacji zastosowanych w tej pracy mają zupełńie inny wpływ na wydajnopść algorytmu w zależności od modelu GPU. OpenCl daje jednak możliwość odczytania wielu parametrów specyfikacji użądzenia, co daje możliwość sparametryzowania algorytmu cechami użądzenia. Najprostszym przykładem takiej parametryzacji jest tradycyjna równoległa pętla for, parametrem w tym przypadku jest maksymalna liczba równoległych wątków dostępnych na użądzeniu.
\linebreak

Tak jak w przykładzie tej pracy iWlość jednocześnie przetwarzanych punktów została uzależniona od dostępnej pamięci prywatnej i współdzielonej.
\linebreak

W innych przypadkach o wyborze optymalnej impementacji może decydować np. dostępność operacji wektorowych o odpowiednim rozmiarze, przyspieszenie sprzętowe dla funkcji takich jak sqrt, sin, log.

Sparametryzowany w ten sposób algorytm może być wykorzystany w oprogramowaniu uruchamianym na komputerze klienta. Wtedy bez względu na posiadany przez niego sprzęt będzie działało wydajnie.
\linebreak



