
%  Ogólnie o zastosowaniu GPU to ogólnych obliczeń


%Karty graficzne początkowo służyły tylko do wyświetlania obrazu na ekranie komputera.
Karty graficzene od kilku lat nie służą już tylko do wyświetlania obrazu na ekranie monitora. Renderowanie coraz bardziej szczegółowej grafiki 3D wymusiło ogromny wzrost ich mocy obliczeniowej, a dodanie możliwości przeprogramowania potoku renderowania umożliwiło zastosowanie tych układów do innych celów.

%(?? Początkowo programiści zmuszeni byli do umieszczania swoich programów w tzw. shaderach.   ---  może by rozwinąć ten temat ??). 
%Co prawda na początku 
Początkowo programiści byli zmuszeni do implementacji algorytmu w postaci tzw. shader'a, gdzie dostępne były tylko operacje typowe dla  przetwarzania grafiki 3D. Dane i wyniki obliczeń musiały być przekazywane jako tablice wierzchołków.

Z czasem jednak powstały platformy, które umożliwiają wykonanie na GPU dowolnego algorytmu. Pierwszą z nich było CUDA (ang. Compute Unified Device Architecture) opracowane przez NVIDIA Corporation w 2007 r. Niestety nie jest ona kompatybilna z GPU innych producentów. 
Dwa lata później powstał pierwszy otwarty standard - OpenCL (ang. Open Computing Language). ?? Co by tu jeszcze dopisać? o tym że działa na prawie każdym GPU, systemie, a nawet na FPGA ?? \linebreak

Obecnie karty graficzne znajdują zastosowanie wszędzie tam, gdzie potrzebna jest duża moc obliczeniowa, a szczególnie gdy trzeba wykonać te same obliczenia na wielu zestawach danych.
Są to zagadnienia takie jak: prognozowanie pogody, symulacje częsteczek, ciał niebieskich, metody Monte Carlo i wiele innych naukowych obliczeń. \linebreak %wymieniać więcej???





Jednym z takich zagadnień są symulacje punktów materialnych. 
Każdy z tych punktów jest osobnym zestawem danych, na którym trzeba wykonać jednakowe obliczenia, co ułatwia stworzenie równoległego algorytmu bez zbyt częstej synchronizacji.
%więc stosunkowo łatwo jest napisać algorytm dla dużej ilości wątków, który nie potrzebuje zbyt częstej synchronizacji.


?? Pisać tu o czym jest praca ??
%Niniejsza praca skupia się na jednym z takich zagadnień, a mianowicie problemie N-Ciał. \linebreak


%Są to między innymi obliczenia fizyczne, symulacje białek, prognozowanie pogody, metody Monte Carlo i wiele innych. 
%Aplikacje wykorystujące GPU ciągle jednak nie są zbyt powszechne ???
%?? Symulacje fizyczne łatwo się zrównolegla oraz implementuje na GPU, bo trzeba w nich wykonać dużo jednakowych obliczeń na kolejnych danych. 

