
%  Ogólnie o zastosowaniu GPU to ogólnych obliczeń


Karty graficzne początkowo służyły tylko do wyświetlania obrazu na ekranie komputera.
Karty graficzene od kilku lat nie służą już tylko do wyświetlania obrazu na ekranie monitora. Renderowanie coraz bardziej szczegółowej grafiki 3D wymusiło ogromny wzrost ich mocy obliczeniowej, a dodanie możliwości przeprogramowania potoku renderowania otworzyło możliwość zastosowanie tych układów do innych celów.
Początkowo programiści zmuszeni byli do umieszczania swoich programów w tzw. shaderach. 
Z czasem powstały platformy takie jak CUDA, które umożliwiają wykonanie na GPU dowolnego algorytmu. 

Obecnie karty graficzne znajdują zastosowanie wszędzie tam, gdzie potrzebna jest duża moc obliczeniowa, a szczególnie do wykonywania tych samych obliczeń na wielu zestawach danych.
Są to zagadnienia takie jak: prognozowanie pogody, symulacje częsteczek, ciał niebieskich, metody Monte Carlo i wiele innych. \linebreak %wymieniać więcej???





Jednym z takich zagadnień są symulacje punktów materialnych. 
Każdy z tych punktów jest osobnym zestawem danych, na którym trzeba wykonać jednakowe obliczenia, co ułatwia stworzenie równoległego algorytmu bez zbyt częstej synchronizacji.
%więc stosunkowo łatwo jest napisać algorytm dla dużej ilości wątków, który nie potrzebuje zbyt częstej synchronizacji.

Niniejsza praca skupia się na jednym z takich zagadnień, a mianowicie problemie N-Ciał.\linebreak


%Są to między innymi obliczenia fizyczne, symulacje białek, prognozowanie pogody, metody Monte Carlo i wiele innych. 
%Aplikacje wykorystujące GPU ciągle jednak nie są zbyt powszechne ???
%?? Symulacje fizyczne łatwo się zrównolegla oraz implementuje na GPU, bo trzeba w nich wykonać dużo jednakowych obliczeń na kolejnych danych. 

