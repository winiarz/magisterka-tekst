
\section { Implementacja na GPU - wersja podstawowa }

\subsection{ Framework OpenCL }

OpenCL jest technologią, która pozwala wykorzystać karty graficzne do wykonywania dowolnych obliczeń. Składa się z dwóch części. Pierwszą jest biblioteka służąca do kontrolowania pracy kerneli. Jest ona dostępna dla wielu języków programowania, między innymi C, C++, Java, Python. Drugą częścią jest język oparty na języku C, który służy do pisania kerneli.
\linebreak
OpenCL został zaimplementowany dla najpopularniejszych współczesnych platform sprzętowych i systemowych. Jest dostępny dla urządzeń producentów takich jak: Nvidia, Amd, Intel, systemów operacyjnych: Windows, Linux, OSX, a nawet dla użądzeń mobilnych pracujących pod kontrolą systemu Android.

\subsection{ Implementacja po stronie host'a }

Urządzenia takie jak karta graficznie nie są przystosowane do samodzielnej pracy. Ich używanie wymaga programu dla procesora CPU, który pełni rolę kontrolera dla kerneli.

\subsubsection{ Wybór i przygotowanie kontekstu urządzenia }
Pierwszym krokiem programu jest pobranie informacji o dostępnych urządzeniach. OpenCL pozwala na sprawdzenie informacji takich jak: nazwa urządzenia, dostępna pamieć, maksymalna ilość wątków oraz lista wspieranych rozszerzeń. Informacje te pozwalają na wybranie odpowiedniego urządzenia z listy.

\subsubsection{ Kompilacja kernela }
Kod binarny dla różnych urządzeń nie musi być taki sam. Aby kernel mógł działać na nich wszystkich można go skompilować dopiero po wybraniu urządzenia, a zatem w czasie uruchomienia programu. 

\subsubsection{ Rezerwacja pamięci na urządzeniu }

W OpenCL rezerwacja pamięci globalnej możliwa jest tylko po stronie host'a. Niezbędna jest rezerwacja pamięci zarówno dla danych testowych, jak i dla wyników obliczeń.

\subsubsection{ Kopiowanie danych testowych na urządzenie }
 
Kernel nie ma dostępu do jakichkolwiek zasobów poza urządzeniem ( dysków twardych, sieci, pamięci RAM ). Wszystkie dane do obliczeń muszą być kopiowane przed uruchomieniem kernela.

\subsubsection{ Uruchomienie kernela }

Uruchomienie kernela sprowadza się do ustawienia jego argumentów i dodania do kolejki. OpenCL automatycznie pobiera kolejne pozycje z kolejki i je uruchamia. Później blokująca funkcja 'clFinish' pozwala na oczekiwanie na zakończenie obliczeń.


\subsubsection{ Kopiowanie wyników z urządzenia }
Po zakończeniu obliczeń następuje kopiowanie wyników z powrotem do pamięci RAM. Ten krok nie jest konieczny w niektórych przypadkach. Np. gdy będą one używane do następnych obliczeń. Mogą one też służyć do wyświetlenia na ekranie dzięki współpracy z OpenGL. 


\subsection{ Podstawowa implementacja po stronie device }

Punkty zostają podzielone między wątki za pomocą pierwszej pętli for. Obliczenia dla każdego z punktów modyfikują tylko jego poprzednią pozycję ( zastępując ją nową ), której nie wykorzystuje żaden inny wątek. Dzięki tej prostej zasadzie nie ma konieczności wprowadzania jakiejkolwiek synchronizacji między wątkami.
\lstinputlisting[language=C]{Source/podstawowa.c}

